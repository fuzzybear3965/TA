\section*{Exercise 5.1}
\subsection*{Problem Statement}
\hfill
\begin{center}
    \fbox{
\begin{minipage}{.85\textwidth}
    A piece of material is uniformly
    magnetized and is placed in proximity of a long straight wire in which
    passes a constant current. Compute:
    \begin{enumerate}[a)]
        \item The force and mechanical moment exerted by the magnetic field on
            the object.
        \item The work that the magnetic field must exert to change the
            orientation of the magnetic moment of the object.         
    \end{enumerate}
    The data of the problem are:
    \begin{enumerate}
        \item $\mu_0 = \SI{4 \pi e-7}{\tesla\meter\per\ampere}$
        \item The volume of the object: \SI{1}{\cm\cubed}
        \item The magnetization of the object:
            \SI{5e7}{\joule\per\tesla\per\meter\cubed}
        \item The current in the wire: \SI{100}{\ampere}
        \item The distance between the object and the wire: \SI{.5}{\meter}
            (consider this value to be much greater than the size of the object)
        \item The angle between the magnetization vector and the magnetic field
            generated by the wire: \SI{0}{\degree}
    \end{enumerate}
\end{minipage}
}
\end{center}

\subsection*{Solution}
As compared to the distance from the wire the object is very small so we can
consider it to be an ideal dipole. We obtain the required amount using the
energy of interaction between the magnetic field generated by the current in the
wire and the dipole moment of the object. This is given by:

\[ 
    U = -\vec{m} \cdot \vec{B}
\]

The field has the form:

\[ 
    B = \mu_0 \frac{I}{2\pi r}
\]

and has the direction (as indicated in the figure) tangent to the circumference
coaxial with the wire. And, by considering the definition of the magnetic moment
per volume (realizing that it's constant everywhere in the volume, we have:

\[ 
    m = \iiint \text{Magnetization}~dV = \text{Magnetization} \cdot V 
\]

Then, using the fact that the angle between $\vec{B}$ and $\vec{m}$ is $\theta$
we have:

\[ 
    U(r,\theta) = - \mu_0 \frac{\text{Magnetization} V I}{2 \pi r} \cos \theta 
\]

Now, the force and the mechanical moment (torque) are:

\[ 
    F(r,\theta) = - \frac{\partial U}{\partial r} \vec{u}_r
\]

\[ 
    \tau(r,\theta) = \vec{m} \times \vec{B} 
\]

We obtain:

\[
    F(r,\theta) = -\mu_0 \frac{\text{Magnetization}~V I}{2 \pi r^2} \cos \theta
    \vec{u}_r
\]

\[ 
    \tau(r,\theta) = \mu_0 \frac{\text{Magnetization}~V I}{2 \pi r} \sin\theta
    \vec{u}_x
\]

The force is attractive and is directed radially, while the torque is directed
along the direction of the wire and determines the orientation of the object
along the lines of force of the field. We note, furthermore, that at $\theta =
0$ the torque is cancelled and only the force acts while at $\theta = \pi/2$ the
opposite happens.

As for the work that the magnetic field performs in order to rotate the object,
from the definition of work we have:

\[
    L_{field} = - \Delta U = U(r,0)-U(r,\pi) = -\mu_0
    \frac{\text{Magnetization}~V I}{\pi r}
\]

Plugging in our numbers we find that $L_{field} = -\SI{4e-3}{\joule}$.

Note that we must perform positive work to combat the work being done by the
field if we seek to keep the dipole moment parallel and in the same direction as
$\vec{B}$. 
