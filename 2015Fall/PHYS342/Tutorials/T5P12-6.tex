\section*{Exercise 5.2}
\subsection*{Problem Statement}
\hfill
\begin{center}
    \fbox{
\begin{minipage}{.85\textwidth}
An electromagnet is made of an iron ring-shaped torus of which a slice is
removed so as to obtain a region in air in which is produced a field that will
be used. Around the torus a wire is wrapped N times and carries current i. If
$l_1$ is the distance between the pole pieces and the average length of the air
gap is $l_2$ then determine the value of the magnetic field in the air. (Use the
fact that the radius of the coils is small compared to the radius of the torus).
\end{minipage}
}
\end{center}

\subsection*{Solution}

Calling $H_1$ the field strength in the air and $H_2$ the field strength in the
iron we have from Ampere's law that:

\[ 
    \oint H \cdot dl = I  = H_1 l_1 + H_2 l_2  = N i
\]

If the radius of the coils is small compared to that of the torus then we can
assume that $H_1$ and $H_2$ represent the value of H at any point, respectively,
in the air and in the iron. To obtain $H_1$ we need another relationship between
$H_1$ and $H_2$. This comes from the continuity of $\vec{B}$.

\[
    \vec{B}_1 = \vec{B}_2
\]

Thus, we can write:

\[ 
    \mu_0 H_1 = \mu_0 \mu_2 H_2 
\]

Finally:

\[ 
    H_1 = \frac{N i}{l_1 + l_2/\mu_2} 
\]

This is the value of the $H$ field in the air gap.
