\documentclass{article}
\usepackage[]{amsmath}
\begin{document}
IPA and two large Kimtechs.
\section*{Lectures 1 and 2: PHYS242 Information}

\[
        \begin{array}{|c|c|c|}
         \hline & \text{Perpendicular Boundary Conditions} & \text{Parallel
    Boundary Conditions} \\ \hline
    \text{Electric Field Bdry Cdtns} & \left( \vec{E}_1 - \vec{E}_2 \right) \cdot \vec{n} =
        \frac{\sigma}{\epsilon_0} & \left( \vec{E}_1 - \vec{E}_2 \right)
        \cdot \vec{t} = 0 \\ \hline
        \text{Magnetic Field Bdry Cdtns} & \left( \vec{B}_1 -\vec{B}_2 \right) \cdot \vec{n} = 0 & \left(
    \vec{B_1} - \vec{B_2} \right) \cdot \vec{t} = \mu_0 \vec{J} \\ \hline
\end{array}
\]

\section*{Lecture 3: Introducing the Vector Potential}

\[ \vec{A} = \frac{\mu_0}{4\pi} \int\limits_{current} \frac{\vec{J}}{r}
dV \]

\[
        d\vec{B}(P) = \frac{\mu_0}{4\pi}I dl \frac{\vec{t}\times
        \vec{r}_{QP}}{r_{QP}^3}
\]

\section*{Lecture 4: Vector Potential as Electrostatics}
Solving for the vector potential can be thought of as solving for the
electrostatic potential where the current is replaced with a volume
charge density : $ c^2 \rho \rightarrow J$.

\section*{Lecture 5: Relativity of Fields}
Relativity of motion determines that B fields change into E fields and
vice-versa. Charge value is independent of reference frame but volume
(change density) is not.

\section*{Lecture 6: Deriving the Weak Formulation}
Electrostatic energy can be written as 
\[ 
        U_e = \int\limits_{\text{all space}} \rho \phi dV -
        \frac{1}{2}\epsilon_0\int\limits_{\text{all space}} ||\nabla \phi||^2 dV
\]

It can be shown that the solution for the right $\phi$ is determined by
minimizing $U_e$ over all $\phi$ (given some $\rho$). 

\section*{Lecture 7: Using the Weak Formulation}
In this lecture we derive an approximation of the capacitance between a
cylindrical conductor and another concentrically spaced conductor. Study the
capacitance example. Try to extend it to third order.

\section*{Lecture 8: Mutual Inductance}
Conducting loops adjacent to one another will exhibit mutual inductance between
both of them. 

\begin{align*}
    M_{21} &= \frac{\Phi_{21}}{I_1} = \frac{\mu_0}{4\pi} \oint_{\gamma_2}
    \left(\oint_{\gamma_1} \frac{\vec{t}_1}{r}~dl_1\right) \cdot \vec{t}_2~dl_2
    &= \frac{\Phi_{21}}{I_1} = \frac{\mu_0}{4\pi} \oint_{\gamma_2}
    \oint_{\gamma_1} \frac{\vec{t}_1 \cdot \vec{t}_2}{r}~dl_1 ~dl_2
\end{align*}

\section*{Lecture 9: Self-Inductance of a Loop using Expansions}
This lecture is particularly involved with regards to the calculations involved.
I would not recommend spending much time on it to study for the midterm.

\section*{Lecture 10: Electrostatic Forces and their Relationship to Work}
The particulars of how to calculate the electric force on a conductor is
supplied, in general. That is:

\[ 
        F_r = - \frac{\partial U_e}{\partial r}\Big|_q = \frac{\partial
        U_e}{\partial r}\Big|_\phi
\]

The special case of the electric force acting on a condenser is also considered.

\section*{Lecture 11: Work and Forces with Currents}

The force acting on a current-carrying wire in a magnetic field is considered.
\[ 
        \vec{F} = \int_{\tau} \left( \vec{J} \times \vec{B} \right) dV
\]

This expression was used to calculate the force between straight wires carrying
current.

The energy required to move a conductor was considered.

\section*{Lecture 12: Multiple Expansions of Electrostatic Potential}

The multipole expansion of a potential was defined in terms of Legendre
polynomials.

The torque on a dipole in an electrostatic field is considered.

\section*{Lecture 13: Dipoles}
The potential energy of a dipole interacting with an electric field is
considered. 

The electrostatic potential due to a configuration of dipoles is considered.

Finally, the surface charge density, $\sigma$, and volume charge density,
$\rho$, of charge given by a distribution of dipoles is given.

Namely: $ \sigma = \vec{P}\cdot \vec{n} $ and $ \rho = -\nabla \cdot \vec{P} $ 

\section*{Lecture 14: Multipole Expansion of Vector Potential}
Multipole expansion of the vector potential is considered. This is worth
reviewing.
\end{document}
