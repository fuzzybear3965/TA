\documentclass{article}
\usepackage[]{amsmath}
\begin{document}
IPA and two large Kimtechs.
\section{Lectures 1 and 2}

\[
        \begin{array}{|c|c|c|}
         \hline & \text{Perpendicular Boundary Conditions} & \text{Parallel
    Boundary Conditions} \\ \hline
    \text{Electric Field Bdry Cdtns} & \left( \vec{E}_1 - \vec{E}_2 \right) \cdot \vec{n} =
        \frac{\sigma}{\epsilon_0} & \left( \vec{E}_1 - \vec{E}_2 \right)
        \cdot \vec{t} = 0 \\ \hline
        \text{Magnetic Field Bdry Cdtns} & \left( \vec{B}_1 -\vec{B}_2 \right) \cdot \vec{n} = 0 & \left(
    \vec{B_1} - \vec{B_2} \right) \cdot \vec{t} = \mu_0 \vec{J} \\ \hline
\end{array}
\]

\section{Lecture 3}

\[ \vec{A} = \frac{\mu_0}{4\pi} \int\limits_{current} \frac{\vec{J}}{r}
dV \]

\[
        d\vec{B}(P) = \frac{\mu_0}{4\pi}I dl \frac{\vec{t}\times
        \vec{r}_{QP}}{r_{QP}^3}
\]

\section{Lecture 4}
Solving for the vector potential can be thought of as solving for the
electrostatic potential where the current is replaced with a volume
charge density : $ c^2 \rho \rightarrow J$.

\section{Lecture 5}
Relativity of motion determines that B fields change into E fields and
vice-versa. Charge value is independent of reference frame but volume
(change density) is not.

\section{Lecture 6}
Electrostatic energy can be written as 
\[ 
        U_e = \int\limits_{\text{all space}} \rho \phi dV -
    \frac{1}{2}\epsilon_0\int\limits_{\text(all space}} ||\nabla \phi||^2 dV
\]

\end{document}
