\section*{Exercise 5.2}
\subsection*{Problem Statement}
\hfill
\begin{center}
    \fbox{
\begin{minipage}{.85\textwidth}
    A cylindrical bar magnetized along its axis is free to rotate about its
    center. It is placed a distance a = \SI{2}{\centi\meter} from a wire
    carrying a constant current (cross sectional area =
    \SI{2}{\square\milli\meter}, length = \SI{3}{\centi\meter}). What is the stable equilibrium position of the
    rod? How large is the magnetic force acting on the rod in this position if
    the current in the wire is I = \SI{10}{\ampere} and the magnetization
    (magnetic moment per unit volume) of the rod is M =
    \SI{300}{\ampere\per\meter}?
\end{minipage}
}
\end{center}

\subsection*{Solution}

We will approximate the rod as an ideal magnetic dipole. Then, we will calculate
all of the variables referring to the position of the center of the bar and
associating with it the overall dipole moment.

\[ 
    \vec{m} = \vec{M} V = \vec{M} \cdot L \cdot S 
\]

Having set the distance from the wire, the equilibrium position of the rod is
obtained by canceling the mechanical torque $\tau$ determined by the
interaction between the magnetic moment of the rod and the field B produced by
the current flowing in the wire. So,

\[ 
    \tau = \vec{m}\times \vec{B} 
\]

By the symmetry of the problem and the solenoidal property of $\vec{B}$ the
magnetic field will be tangent to the circumference orthogonal to the wire and
concentric with it. In order that $\tau = 0$ we have to cancel the angle between
$\vec{B}$ and $\vec{m}$. In order that the equilibrium position is stable then
it needs to be a minimum of the potential energy of the system. We can write
this as:

\[ 
    E = - \vec{m} \cdot \vec{B}
\]

The minimum for this happens when $\vec{m}$ is the same direction as $\vec{B}$.
Using Ampere's law, we can write:

\[ 
    B = \frac{\mu_0 I }{2 \pi r} 
\]

Substituting our expressions for $B$ and $m$ allows us to write:

\[ 
    E = -M S L \frac{\mu_0 I }{2 \pi r} 
\]

To calculate the force, we need to take the derivative of this expression at
$r=a$.

\[ 
    F = \frac{d E}{d r}\big|_{r=a} = MSL \frac{\mu_0 I}{2 \pi a^2}
\]

The direction is given by that which joins the wire and the rod. Plugging in
numbers yields \SI{9e-8}{\newton}.

To get the exact solution we need to consider separately the infinitesimal
elements of the rod and integrate along its length. Consider the rod as
arranged in the figure, in the equilibrium position. Given the form of B that we
obtained earlier, r is now:

\[
    r = \frac{a}{\cos\theta}
\]

The energy of a single element is:

\[ 
    e = - \vec{m}\cdot \vec{B} = -\vec{M} dV \cdot \vec{B} = - \vec{M} \cdot
    \vec{B} S dx
\]

If we take into account the angle between the dipole and the field:

\[ 
    e = -M B \cos\theta S dx 
\]

But since $dx = d(a \tan\theta) = \frac{r}{\cos \theta}d \theta$ we have:

\[ 
    e = - \frac{M S \mu_0 I}{2 \pi} d \theta
\]

Integrating now over the whole rod will give us the total energy:

\[ 
    E = \frac{M S \mu_0 I}{\pi} \alpha 
\]

where $\alpha = \arctan \frac{L}{2 a}$.

To get the force we need to take the derivative of E with respect to r and
evaluate it at $r=a$. Doing this yields:


\[ 
    F = \frac{\partial E}{\partial r}\big|_{r=a} = \frac{M S L \mu_0 I}{2 \pi}
    \frac{1}{a^2 + (L/2)^2}
\]

Plugging in numbers yields: $F = \SI{5.76e-8}{\newton}$.
