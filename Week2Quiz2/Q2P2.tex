\begin{homeworkProblem}[Quiz 2, Problem 2] 
    Okay, the problem of solving for the z-component of the electric
    field generated by a charged disk seemed to be difficult for you
    guys. Let me discuss a few ways in which this problem could be
    solved.
    
    The first approach that I will use will be to establish the
    electric field set up by a ring centered around the z-axis. Then, I
    will integrate over an infinite number of rings whose radius varies
    smoothly from 0 to the radius of the disk.

    The second approach that I will use will be to establish the
    electric field from a disk directly using polar coordinates. I will
    integrate around the disk (in the angular direction) and I will
    integrate radially outward (from 0 to R, the radius of the disk).
    This involves the use of the differential area alement $\diff a$ in
    polar coordinates. I will try my best to explain the origins of this
    differential area element. That may go in an appendix at the end.

    \begin{homeworkSection}{Approach 1}
        Okay, so I'm going to first start by solving for the electric
        field established by a uniformly charged ring. The ring will
        have a constant charge per unit length, $\lambda$, such that the
        total charge on the ring is $2\pi R \lambda = Q$. My approach
        will be to chop up the ring into a bunch of infinitesimal point
        charges $dq'$ and evaluate the electric field, $\diff \vec{E}$,
        generated by each of these infinitesimal point charges. Once I
        have this contribution to the electric field I will integrate
        over all the infinitesimal point charges to find the net
        electric field $\vec{E} = \int \diff \vec{E}$. 

        The key to starting to write this integral is to realize that
        each infinitesimal check of charge $dq'$ is like a little point
        charge located at position $r'$ (see figure \ref{fig:geom.eps})
        and, as such, the electric field contribution due a particular
        chunk of charge, $dq'$, looks like:

        \[ \diff \vec{E}(\vec{r}) = \frac{k \diff q'}{|\vec{\scriptr}|^2}
        \hat{\scriptr} \]

    \end{homeworkSection}
\end{homeworkProblem}
