\begin{homeworkProblem}[Problem 23.36]
	\textbf{Consider the electric dipole shown in Figure P23.36. Show that the electric field at a distant point on the +x axis is $E_x \approx 4k_e q a /k^3$.}

	Okay, the trick with this question is to understand what ``distant point'' means. What we mean is that the distance d that we are away from the origin along the positive x-axis is much much greater that 2a, the distance between the two charges. So, let's consider the ratio of the $\frac{2a}{d}$. Well, if $\frac{2a}{d}$ is small, then $\frac{a}{d}$ is even smaller. So, we can really see if our answer has anything that looks like a $\frac{a}{d}$ in it at the end. If it does then we will expand about a small $\epsilon \propto \frac{a}{d}$ using a Maclaurin series expansion (e.g. a Taylor series expansion about zero).

	First we need to find the electric field everywhere on the positive x-axis. Then, we need to see if we can expand. The electric field everywhere on the positivie x-axis is just given by the sum of the electric field contributions from the positive and negative charge:

	\begin{align}
		\label{}
		\vec{E}(x>0) &= k q \frac{\hat{x}}{(d-a)^2} + k q \frac{-\hat{x}}{(d+a)^2} \nonumber \\
		\intertext{I didn't use my standard method of writing $\vec{r}$, $\vec{r'}$ and $\vec{\pmb{\scriptr}}$ because it's a one dimensional problem so it's pretty easy to set up this equation. The denominator is the distance from the respective charge and the top encodes information about the strength of the charge and the direction of the electric field. Note that for x<0 this equation does not hold because the term that has (d+a) in the denominator would reference the positive charge. It's electric field would point in the positive $\hat{x}$ direction. Notice, also, that this function is not valid for anywhere off axis since then the electric field would have both an x and a y component.} \nonumber \\
		&= k q \hat{x} \big( \frac{1}{d^2(1-(a/d))^2} - \frac{1}{d^2(1+(a/d))^2} \big) \nonumber \\
		& = \frac{k q \hat{x}}{d^2} \big( \frac{1}{1-\epsilon^2} - \frac{1}{1+\epsilon^2} \big) \nonumber
	\end{align}
	Now, we have exactly the kind of expression we were looking for. Let's expand $f^-(\epsilon) = \frac{1}{1-\epsilon^2}$ and $f^+(\epsilon) = \frac{1}{1+\epsilon^2}$.

	Expanding first $f^+(\epsilon)$:

	\begin{align}
		\label{}
		f^+(\epsilon = 0) &\approx f(0) + \frac{d}{d\epsilon}{f^+(\epsilon)}|_{\epsilon=0}\epsilon + \frac{d^2}{d\epsilon^2}{f^+(\epsilon)}|_{\epsilon=0} \frac{\epsilon^2}{2!} + \frac{d^3}{d\epsilon^3}{f^+(\epsilon)}|_{\epsilon=0} \frac{\epsilon^3}{3!} \cdots \nonumber \\
		\intertext{At this time I encourage you to evaluate these expressions yourself. I ask you to excuse me for saving myself some time.} \nonumber \\
		&= 1 - 2\epsilon + 3\epsilon^2 - 4\epsilon^3 + \cdots \nonumber \\
	\end{align}

	$f^-(\epsilon = 0)$ can be obtained in the exact same manner. However, since $f^-(\epsilon)$ only differs from $f^+(\epsilon)$ by a minus sign on $\epsilon$ we can easily write the expansion for $f^-(\epsilon)$.

	$f^-(\epsilon) \approx = 1 + 2\epsilon + 3\epsilon^2 + 4\epsilon^3 + \cdots $. Now, what we had in our earlier expression was $f^-(\epsilon) - f^+(\epsilon) \approx (1 + 2\epsilon + 3\epsilon^2 + 4\epsilon^3) - (1-2\epsilon + 3\epsilon^2 - 4\epsilon^3) = 4\epsilon +8\epsilon^3 + \cdots$. 

	Now, if $\epsilon$ is sufficiently small then $8\epsilon^3$ looks much smaller than $4\epsilon$. So much so that this smaller term is not worth considering. Thus, to first order in $\epsilon$ (that is, to the first power in $\epsilon$) our electric field at distant points is:

	$\vec{E}(x>0) = \frac{k q \hat{x}}{d^2} 4 \epsilon$. But, what is $\epsilon$? It's nothing more than $\frac{a}{d}$. Substituting this, now: $\vec{E}(x>0) = \frac{k q a\hat{x} }{d^3}$.

	This isn't quite the form in the problem statement. Now, I have to realize that $d$ is what they are calling $x$. It's the distance I have traveled along the positive x-axis. So, finally, $\vec{E}(x>0) \approx \frac{k q a \hat{x} }{d^3} $. 

	There may be a couple typos in here, but this is the idea.
\end{homeworkProblem}
